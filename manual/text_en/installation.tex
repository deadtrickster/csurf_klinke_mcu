\section{Installation}\label{installation} 

To install the extension download \url{http://tinyurl.com/6664hym}
and unpack the zip--archiv in the
plugin directory of \reaper. Afterwards the folder with the original
{\tt reaper\_csurf.dll} file should also contain the file {\tt
  reaper\_csurf\_mcu\_klinke.dll}, this manual and a new folder named
{\tt MCU}. This {\tt MCU} folder contains default configuration files
and FX maps. Do not modify them. Your own configuration files and FX
maps are stored in the folder {\tt My Documents\textbackslash MCU}
that is created when you start the
\mcu surface for the first time.\\

\noindent
To activate the plugin you need to follow these steps:
\begin{compactitem}
\item open the \reaper preferences
\item scroll down and select ``Control Surfaces'' 
\item if the original ``Mackie Control Universal'' surface was added, remove it
\item add the ``Mackie Control Universal (Klinke)'' surface
% \item optional: increase the ``Control surface display
% update frequency'' in the Control surfaces preferences of \reaper to something between
% 30 and 50 Hz.  
\end{compactitem}

%If you have extenders:
%\begin{itemize}
%\item set the Bank Size parameter in the "Mackie Control Universal (Klinke)"
%preferences dialog to the total number of channels (or maybe to 8 less, in the
%case that you want to use the main unit in non-mixing modes the most time)
%\item remove original "Extender" surfaces and add "Extentder (Klinke)" surfaces
%for each of your extenders
%\item set the Surface Offset parameter in the Surface prefercences dialogs
%corresponding to the order of your setup (the leftmost unit should have an
%offset of 0, the unit on his right side an offset of 8 ...)
%\end{itemize}

\noindent
For the \actionmode it is useful to import Action
shortcuts:\footnote{If you want to assign all actions by yourself
  there is need to import anything, but  the predefined actions can be
  useful to understand the features described in the
  \hyperref[actionmode]{Action-Mode} section.}
 
\begin{compactitem}
\item open the \hyperref[F:actiondialog]{Actions dialog} in
  \reaper(e.g. by pressing the '?' key)
\item press the Import/Export button on the lower left corner and select
  "Import..."
\item in the appearing file-dialog, select the file {\tt \ldots\tbs
    Reaper\tbs Plugins\tbs MCU}\newline{\tt \tbs
    DefaultActions.reaperKeyMap}.
\end{compactitem}

\subsection{Control Surface Settings Options}\label{settings}
There are four options in the Control Surface Settings dialog that can
be activated (the later three are useful for owners of the Behringer
BCF2000 and maybe other controllers as well):
%\pikFigure[][8.05cm][-2.5mm]{Screenshot_CSS}{Control Surface Settings}{}{}
%\vspace{-5mm}
\begin{itemize}
\item Swap left/right arrow on zoom: By default the left arrow key
  zooms out of the arrangement and the right arrow button zooms in (assuming
  zooming is activated via the \zoom button). This can be swapped.
%\item Controller follows track selection: Per default the shown tracks at the
%Mackie Control can be only changed via the Mackie Control itself. If this
%option is turned on and a track that isn't shown at the controller is selected,
%then the shown tracks are adjusted so that the selected track is one of the
%shown tracks.

\item No level meter: Do not show the level-meters in the Pan-Mode,
  always show the dB values instead (the bcfview Application does not
  support the level-meters protocol, so for BCF users it is useful to
  choose this option).
\item Use keyboard modifier: The Behringer BCF does not have 
  \option/\control and \alt modifier buttons. When this option is
  selected, the keyboard modifier can be used instead. The left
  \alt key substitutes the \alt button, the right \alt key the
  \option button.
\item Fake fader touch: The BCF2000 does not have fader touch sensitive
  faders, so after a fader movement the extension simulates that the
  fader being touched for an additional 2 seconds.
\end{itemize}


%%% Local Variables: 
%%% mode: latex
%%% TeX-master: "../mcu_klinke_manual"
%%% End: 
