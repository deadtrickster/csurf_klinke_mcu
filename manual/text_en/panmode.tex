\subsection{Pan-Mode}\label{panmode} 
The extended version of the behaviour of the MCU plugin that comes
with \reaper is called Pan-Mode and can be selected by pressing the \pan
button. The Fader and the \vpots,  \mute and \solo buttons should be
self-explanatory with the exception that double-clicking \solo can be
used to solo the track exclusively. The other functions are:

\begin{itemize}
\item \vpots: Change the pan of the track

\item \rec: Arm Track on/off
	\bemod
	\begin{itemize}
	\item \shift: Monitor on/off. As long as \shift is pressed the
          \solo button LEDs show the Monitor status (On = monitoring
          is active)
 	\item \option: Rec Mode Input/None. As long as \option is
          pressed the \solo button LEDs are show the Rec Mode
          status (On = Rec Mode is set to Input, Off = all other
          settings)
	\end{itemize}
	
\item \select: Select a track and deselect the other tracks
	\bemod 
	\begin{itemize}
	\item \control: Add the track to the tracks already selected.
	\item \shift: Select all tracks from the selected track to the
          track selected before (via the MCU). So \control \& \shift
          has the same behaviour as using the Control \& Shift key
          while selected track using the mouse in \reaper.
	\item \alt: Same as \shift, but only the tracks that are
          accessible via the MCU are selected. See section
          \hyperref[trackfilter]{Trackfilter} for details.
	\item press and hold: If the \hyperref[foldermode]{Folder
            mode} is activated and the track has children, use the
          track as root track. See \hyperref[foldermode]{next
            section} for details.
	\end{itemize}

\item \bankdu: Show the previous/next eight\footnote{When
\anchors are activated: the previous/next eight tracks minus the
number of \anchors.} tracks on the MCU

\item \channeldu: Show the previous/next track on the MCU

\item \gv: If the \hyperref[foldermode]{Folder mode} is activated and the
track has a parent track the LED is switched on and when the button is pressed,
the parent track is used as root track. See \hyperref[foldermode]{next
section} for details.

\item \flip: Flip volume and pan (use the \faders to change the pan, and the
\vpots to change the volume)

\item \mfader: Change the master volume
\end{itemize}

\noindent

\subsubsection{Folder-Mode}\label{foldermode}
By default the folder structure of \reaper is not reflected on the
MCU. However, it is possible to change this via the Folder-Mode
\attribute in the \hyperref[T:multitrack_options1]{Pan-Mode
  Options}. When the Folder mode is activated, the tree-like structure
of the tracks in \reaper is reflected on the MCU and you only see the
children of a track and, if desired, the parent of these children. The
idea behind this is that the folder structure corresponds to
sub-mixes, and you can easily select and work on a single sub-mix or
go back a step and mix only on the level of the sub-mixes. This can
improve track navigation a lot.\footnote{Theoretically, you can create a
  structure with 584 tracks and each track is no more then five clicks
  away. Or in combination with the \hyperref[quickjump]{Quick Jump}
  feature, no more then three click.} When you switch to Folder-Mode, only
tracks are shown that have the Master track as their root. For all
tracks that have children, the LED below the \vpot is illuminated. You can
select these tracks as a new root track by pressing the
corresponding \select button for longer than one second. To move back to
the root track, press the \gv button.

\subsubsection{Editor}\label{panmodeeditor}
Before other concepts are introduced, it is useful to take a look at the
\hyperref[F:Screenshot_Track_Mode]{Track Settings Editor} in
Figure~\ref{F:Screenshot_Track_Mode}.\footnote{To safe space, the screenshot
displays less tracks then the real editor.} You can open this editor by pressing
\alt \pan. 

\pikFigure[0.75][13.75cm][-2.2mm]{Screenshot_Track_Mode}{Track settings editor
(\alt \pan)}{}{}

The table rows represent the tracks of the current project. The
leftmost column shows the \cs that controls the track (if any). In the
``Nr.'' column the track number as defined in \reaper is shown. TCP and MCP
are equal to the ``Show tracks in track list'' and ``Show tracks in
Mixer'' settings in \reaper. The other columns are explained in the
following sections.

\subsubsection{Track Names}\label{tracknames} 
Before v0.8, the name shown on the MCU was just the first six characters of the
track name in \reaper. This has been  improved, the name is now derived using the
following rules:

\begin{compactitem}
\item When a name is given in the Display column, this name is used
  for the MCU display
\item When no name is given but the track name contains the character
  '$\mid$', the first six characters after the $\mid$ are used
\item If neither is the case, the first six characters of the track
  name are used
\end{compactitem}


\subsubsection{Track Filter}\label{trackfilter}
With the ``Show'' \attribute in the
\hyperref[T:multitrack_options1]{Main Options} you can limit the
MCU-accessible tracks to different sets. Some of these
sets are defined via the \reaper GUI, like tracks
with sends, or tracks that are also shown in the \reaper mixer. However,
you can also define a \mcu specific set, so that the tracks 
shown on the MCU are independent of any setting in the \reaper
GUI. To do this, set  Show \attribute to ``MCU Set'' in the
\hyperref[F:Screenshot_Track_Mode]{Track settings editor} and select
the tracks that should be accessible in the ``MCU'' column.

\subsubsection{Anchors}\label{anchor}

A \cs can be attached to a track, so this track can always be
controlled via the same fader etc., independent of the track
navigation via the \bankdu and \channeldu buttons. To attach a track,
select one of the eight anchors in the Anchor column of the
\hyperref[F:Screenshot_Track_Mode]{Track Settings Editor}. For example
in the screenshot the track number 9 ``voc'' is always accessible through
\cs 2. This is also the case when the \foldermode is active, so that
anchor tracks are shown on the MCU even when they are not part of the 
sub-mix displayed.

I suggest avoiding gaps between anchored Channel Strips because otherwise navigation
throgh tracks can get confusing.

It is possible to turn off all anchors via the ``Use Anchor''
\attribute in the \hyperref[T:multitrack_options1]{Main Options}, in
which case the settings in the
\hyperref[F:Screenshot_Track_Mode]{Track settings editor} will be
ignored.

\subsubsection{Quick Jump}\label{quickjump}

To simplify navigation, you can define up to eight tracks that can be
set to the leftmost \cs (the one that is not used for an Anchor track)
quickly, followed by the other tracks as usual. To activate the Quick
Jump feature, press and hold the \pan button. The display will show
the names of the targets tracks (called Pads). The track can then be
selected by pressing the corresponding \vpot.

The three rightmost columns in the
\hyperref[F:Screenshot_Track_Mode]{Track Settings Editor} are used to define the
Pads for this purpose:

\begin{compactitem}
\item Jump Slot: This column determines the tracks which can be
  selected as leftmost shown tracks.
\item J. Name: If empty, the name displayd will be the same as the
  track name. It can be useful to replace the track name by a
  different name in the Quick Jump display.  E.g. if the track is
  named ``BD'' and the following tracks are named ``SD'', ``Hi Hat''
  etc., you can overwrite the ``BD'' text in this column with
  ``Drums''. When you select the ``Drums'' Jump Slot, you will get the
  ``BD'' track at the leftmost position on the MCU. 
\item Root: This checkbox plays only a role when the \foldermode is
  active and the track has children. If this is the case and the Root
  checkbox is set, the track will be used as the new root track, so
  that all the children will be shown. If the checkbox is not set, the
  track and other tracks on the same level will be shown instead. This
  is also reflected in the MCU display, Root jumps are shown in the
  upper row, ``normal'' jumps in the lower.
\end{compactitem}

\subsubsection{Options}
Table~\ref{T:multitrack_options1} shows the Main Options of the Pan-Mode which
already explained in the sections before.

\pikTable[14.43cm]{multitrack_options1}{Main Options in Pan-Mode and Action-Mode
(\option)}{}{}

The 2nd Options are shown in Table~\ref{T:multitrack_options2}. They allow to
synchronize the tracks in the TCP/mixer to the tracks shown on the MCU, or, in
the caseof the TCP, only the tracks are shown that are selected.

\pikTable[15cm]{multitrack_options2}{2nd Options in Pan-Mode and Action-Mode
(\shift \option)}{}{} 

%%% Local Variables: 
%%% mode: latex
%%% TeX-master: "../mcu_klinke_manual"
%%% End: 
