\section{Introduction}

Welcome to the manual for the \mcu Extension for \reaper (v3.6 or
higher). This extension enhances support for the Mackie Control
Universal (MCU) controller, but can also be used for other controllers
that use the MCU protocol (please read the manual of your
controller to check if and how your controller supports this
protocol). In addition, the extension also supports the Behringer
BCF2000 and BCR2000. tehsux0r has written a setup guide for the
BCF2000 that can be found
\href{http://forum.cockos.com/showthread.php?t=117909}{here}
and nofish wrote a setup guide for the BCR2000 that can be found
\href{http://forum.cockos.com/showthread.php?t=60110}{here}.

\mcu is released under the GPL 2 licence, for details check
section~\ref{disclaimer}.

\subsection{Current Limitations}
\begin{itemize}
\item \mcu only runs under Microsoft Windows, no OS X version exists
  and none is planned from my side. However, the source code is open to
  everyone and most parts are based on the cross platform library
  JUCE, so if anyone wants to try to port the extension to OS X, this
  should be feasible (and I will try to support them as much as
  possible).

\item In the current version the extension does not support the Mackie
  Control Extender (MCE), the changes from v0.64 to v0.8 where too
  far-reaching to keep the extension compatible without owning a
  MCE. 
%I hope that this can be changed in the future (somewhen in
%2011), but i can't guarantee anything. 
  If you have a MCE and think
  it is more important to use the extender than the new features in
  v0.8, you can still download the older v0.6 from here:
  \url{http://tinyurl.com/67qjt6y}.
% No support is planned for the Mackie C4.

\end{itemize}

% \subsection{Manual style}
% 
% Buttons are written in \textsc{Small Caps Shape}. Links (to URL-adresses or
% also to different part of the manual) are marked in blue.
% \todo{vervollstaendigen}

\subsection{New in v0.8}

The most important changes in v0.8:
\begin{compactitem}
\item \hyperref[buttonactions]{Buttonx-emulating actions} (since v0.8.2)
\item \reaper folders can now be reflected on the MCU using the \foldermode
\item \anchors allows locking a \cs to a fixed track
\item \hyperref[fxpresets]{FX presets} (store and recall FX settings) including
a Blind Test feature
\item \hyperref[localmaps]{FX Local maps} (different maps for each FX
instances of the same FX)
\item improved syncing of the MCU state with the \reaper GUI
\item \hyperref[globalactions]{Global Actions} assignments can be named and
shown in MCU display
\item \hyperref[quickjump]{Quick Jumps} added for faster navigation  
\item improved \hyperref[tracknames]{Track Name} handling
\item FX favorites can be stored in \reaper project file
\end{compactitem}

\subsection{Known Issues}
\begin{itemize}
\item On some computers the MCU freezes after  \reaper is started and
  some buttons are pressed. This problem disappears automatically after
  a few minutes or by touching one of the faders several times (until you
  see that the display reacts to the touch, e.g. shows the dB
  value of the track).
\end{itemize}

\subsection{Donations}

Writing the plugin was a lot of work (but also fun most of the time), writing this
manual too (without the fun part). So acknowledgments in the form of donations are
welcome and can be made via PayPal. To donate click
\href{https://www.paypal.com/cgi-bin/webscr?cmd=_s-xclick&hosted_button_id=LR54GZHGL6VHA}{here}
or on the Donate button in the Control Surface
Settings dialog. If your DAW does not have an internet connection, you can also
send the donation via PayPal manually to klinkenstecker@gmx.de. But I am also
delighted when I get positive feedback via email or the \reaper forum ;-)


\subsection{Credits}
I want to thank:
\begin{compactitem}
\item Kathrin for proof-reading the manual and being such a great
  hiking companion
\item Justin for always improving the \reaper API when I need
  additional features (and of course \reaper in general)
\item Chris Doerman (cdoerman) for his Overlay (see
  section~\ref{overlay}), FX maps (e.g. take a look at the hugh Pod
  Farm map, it's an amazing piece of work) and his great help during
  beta-testing
\item Ryan (yagonnawantthatcowbell) and Curvespace for contributing so
  many FX maps
\item axelsk for the Oxford FX maps
\item musicbynumbers for the
  \href{http://forum.cockos.com/showpost.php?p=473466&postcount=127}{Behringer
    BCF2000 setup-guide} and a lot of help on the BCF2000 mode
\item tehsux0r for the even more extensive
  \href{http://forum.cockos.com/showthread.php?t=117909}{Behringer
    BCF2000 setup-guide}
\item nofish for the Behringer
  \href{http://forum.cockos.com/showthread.php?t=60110}{BCR2000
    setup guide}
\item Jason (Drumbum), Miquel (anticlick), Alison and JosMuysers for
beta-testing
\item Jules for the great JUCE library
\end{compactitem}

%%% Local Variables: 
%%% mode: latex
%%% TeX-master: "../mcu_klinke_manual"
%%% End: 
