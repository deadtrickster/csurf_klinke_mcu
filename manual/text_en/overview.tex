\section{Overview}\label{overview}
\vspace{-5mm}
\pikFigure[0.355][][-3mm]{MCU_Overview}{Overview of the MCU}{}{}

Figure~\ref{F:MCU_Overview} shows the MCU hardware and how its elements are
partitioned:
\begin{itemize}

\item The \css is highlighted in blue, it includes the \mfader and the
  \flip, \gv, \nv and FADER BANKs buttons). This section can be set in
  different modes, which allows controlling the \reaper mixer
  (including sends), editing FX parameters and triggering \reaper
  actions. Details are described in section~\ref{csmodes}.
 
\item The F1-F8 buttons (in the read area) can be used to store and
  recall Markers, Time Regions (\option) and Loops (\alt). Pressed in
  combination with the \control modifier, the corresponding element is
  stored. For details see section~\ref{regions}.

\item The buttons in the Global View (green) area can be used to
  trigger \reaper actions. Those assignments are independent of the
  \cs mode.\footnote{Additional actions can be assigned in the
    \hyperref[actionmode]{Action-Mode} of the \css.} Details are
  described in section~\ref{globalactions}.

\item The Transport section (yellow area) highlights all controls that
  can be used to change the \ec, the Loop or the Time Region, and
  those that start/stop playback and recording. Details are described
  in the \hyperref[transport]{following section}.

\item The remaining buttons are described in the \hyperref[misc]{Misc}
  section, apart from the Automation and Utilities buttons which should be
  self-explanatory.

% e.g. with the \jw, the modifier buttons determine what
% should be change. Without any modifier it's the \ec, with \shift or \option the
% Time Region start/end and with \control or \alt the Loop start/end.


\end{itemize}
