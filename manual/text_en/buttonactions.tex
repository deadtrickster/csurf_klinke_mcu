\subsection{Button-emulating actions}\label{buttonactions}

The previous section described how \reaper-Actions can be triggered
by the MCU. But it is also possible to use the \reaper-Action
system to ``trigger'' the buttons of the MCU. Of course the buttons
are only pressed virtually, but for the \mcu extension it doesn't make
any difference if a button is pressed with a finger or a \reaper
action. But what is the use-case for this feature? For owners of a
real Mackie MCU I don't see any, but the owners of other hardware like
the BCF2000 can use this feature to ``extend'' there hardware either
with keys from their computer keyboard or with other MIDI hardware
like the Korg nanoPAD.

\pikFigure[0.73][11.4cm][-2.4mm]{Screenshot_Actions_Klinke}{\reaper-Actions
  added by the extension}{}{}

How does this work? Figure~\ref{F:Screenshot_Actions_Klinke} shows the
action dialog with the action filter set to ``Klinke''. As you can
see, all actions begin with the String ``Mackie Control Klinke'',
followed by the name of a button of the Mackie Control. The action
descriptions end either with ``(button)'' or ``(key)''. You can assign
these actions as described in section~\ref{globalactions}, the only
difference is, that you don't press a button of the Global View
section, but a button on a different MIDI controller or a key on your
computer keyboard. 

So what is the difference between the ``(button)'' and ``(key)''
variants of those actions? As mentioned, an action simulates that
a MCU buttons is pressed. But a MCU button does also send an event
when a button is released, so this must be simulated also. But
e.g. the release of a computer keyboard key isn't recognized by
actions that are assigned to this key. So the ``(key)'' variant
automatically send also the release event directly after the press
event. The ``(button)'' variant waits after the press event till the
action is triggered again and then sends the release event. Most MIDI
controllers sends an event when a button on the controller is released,
in this case they should be assigned to the ``(button)''
variant. Coputer keyboard keys and MIDI controller buttons that only send an
event when the button is pressed should be assigned the ``(key)'' variant.

But fore some buttons like the ``Alt modifer'' only the ``(button)''
variant exists. For those buttons it's important that they are hold
for a longer time, so that the ``(key)'' variant doesn't help. But you
can also assign keys to the ``(button)'' variant, in this case the
button is virtually hold till you press the key for the second time.

