\subsection{Action-Mode}\label{actionmode} 

The button \eq in the VPOT ASSIGN section switches the \vpots to the
Action-Mode, the other elements of the channel stripes will work in
the same way as in the Pan-Mode.

For the Action-Mode it is useful to import Action assignments as described in
the \hyperref[installation]{Installation} section.
%\footnote{If you want to assign all actions by yourself you don't need to
%import anything, but it the predefined actions can be helpful to understand the
%features descriped here.} 
When you have done this, you can use the \vpots in the Action-Mode to
trigger \reaper-Actions, e.g.  moving items etc. How does this work?
The MCU sends different MIDI-CC message to \reaper when pressing a
VPot, releasing it and moving it to left/right in a pressed or
unpressed state, so there are six different CCs per \vpots.  In
combination with the \shift modifier, you get six additional CCs. As
described in the \hyperref[globalactions]{Global Actions} section, you
can assign those MIDI events to \reaper actions.

There are eight banks which can be assigned to different actions (the
imported shortcuts using only four of them). To switch to a different
bank, press and hold the \eq button again while you are in the Action-Mode.
The display will show the name of the banks and you can select one by
pressing the corresponding \vpot.

Of course you can add and change the display text for the banks and
actions, \alt \eq opens the \hyperref[F:Screenshot_Action_Mode]{Action-Mode
editor}.

\pikFigure[0.6][13.3cm][-2.4mm]{Screenshot_Action_Mode}{Action-Mode editor
(\alt \eq)}{}{}


\pikWrapFigure[][6.1cm][-2.5mm]{assignrelativ}{Assignment dialog}{}{}
Additionally the Action-Mode also supports  ``(midi cc only)'' actions. Therefore
you can enable in the \hyperref[F:Screenshot_Action_Mode]{Action-Mode
editor} the ``Relative Mode'', then the CCs that are send to \reaper are
``Relative 2'' instead of ``Absolute''. 


In this case you must set the {\tt MIDI CC:} field in the \reaper assignment
dialog to ``Relative 2'', as shown in Figure~\ref{F:assignrelativ}.

In ``Relative Mode'' you can set the rotation speed via the ``Normal Speed''
and ``Pressed Speed'' settings.

The normal speed is used when the \vpot is not
pressed while rotating, the pressed speed when the \vpot is
pressed. For example this is
used in the imported assignments in the FXparam bank to realize fine adjustment
while pressing the \vpots. The speed is set to 12 in ``Normal Speed'' and 4 in
``Pressed Speed''.
%In Pan-Mode the Quick Jump navigation isn't available directly, because
%pressing the \eq button is used for selecting the action bank. But with a
%double-press on the \pan button, followed by a Quick Jump using a \vpot and
%pressing afterwards the \eq button, you can still jump around.


%%% Local Variables: 
%%% mode: latex
%%% TeX-master: "../mcu_klinke_manual"
%%% End: 
