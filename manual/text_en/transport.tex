\clearpage
\section{Transport}\label{transport}

For the Transport and also the \hyperref[regions]{Regions and Markers} section
the modifier buttons have some common usage, \control is used to store stuff
(this is always the case if something can be stored), \option is used
if the Time Region should be modified\footnote{If there is a difference
between Time Region start and end, \shift is used to select the start and
\option to select the end.} and \alt is used for the Loop selection\footnote{If
there is a distinction between Loop start and end, \option is used to select the
start and \alt to select the end.}. In the event that something
corresponds to the time signature, \shift can be used to select beats instead of bars.
\subsection{\play, \stopp and \cancel}
\begin{itemize}
  \item \stopp: Stop the playback and jump to the \ec
  \bemod
  \begin {compactitem}
    \item while recording: Stop and save all recorded media
    \bebemod
	\begin {itemize}
	  \item \alt: Stop and show Dialog
	\end{itemize}
	\item Double-click: Set the \ec to the nearest bar\footnote{Time
	Signature changes are ignored.}
	\bebemod
	\bebemod
	\begin {compactitem}
		\item \shift: Set the \ec to the nearest beat 
		\item \option: Set Time Selection start/end to the nearest bar
		\item \shift \option:  Set Time Selection start/end to the nearest beat
		\item \alt: Set Loop start/end to the nearest bar
		\item \shift \alt: Set Loop start/end to the nearest beat
	\end{compactitem}
  \end{compactitem}
  \item \cancel: Stop and delete all recorded media
  \item \play: Start the playback
  \bemod
  \begin {compactitem}
    \item while playing: Stop at current play position
    \item \alt: Create a Loop around the \ec with a length of 1 bar and
    play this loop
    \item \shift \alt: Create a Loop around the \ec with a length of 1
    beat and play this loop
  \end{compactitem}
\end{itemize}

\subsection{Jog Wheel}
The Jog Wheel's behaviour can be modifier with the MODIFIER buttons:
\begin{compactitem}
\item \shift: move Time Region start
\item \option: move Time Region end
\item \control: move Loop start
\item \alt: move Loop end
\end{compactitem}

The movement depends on the Zoom level and the status of the \scrub
button. If \scrub is off, the position will move one bar or beat, if
\scrub is on, the position will be moved by the time that is equivalent to two
pixel in the \reaper GUI.

\subsection{\rewind and \forward}

The \rewind \& Fast Fwd (\forward) Button can be set in Mode "Normal", "Marker" or
"Nudge" (note that the Nudge Mode has not been implemented). When \rewind or \forward is
pressed in combination with the \marker or \nudge button, the MCU will
automatically switch back to the original mode after you release the \marker/\nudge
button.\footnote{The same mechanism works for the VPOT ASSIGN buttons in the
\hyperref[csmodes]{\csms}.}

\subsubsection{Normal-Mode}
\begin{itemize}
\item \rewind/\forward: move \ec 1 bar backward/forward. The \rewind/\forward
buttons can be held down for continuous movements.\footnote{The maximum speed of
the movement depends on the Control surface display update frequency which can be set
in the preferences.}
\bemod
\begin{itemize}
  \item \shift: move \ec 1 beat backward/forward 
  \item \control: Goto Project start/end
  \item \option: Goto Time Selection start/end
  \item \shift/\option: Set Time Selection start/end to the \ec position 
  \item \alt: Goto Loop start/end
  \item \shift/\alt: Set Loop start/end to \ec 
\end{itemize}
\end{itemize}

\subsubsection{Marker-Mode}
\begin{itemize}
\item \rewind/\forward: Previous/Next marker
\bemod
\begin{itemize}
	\item \option: Previous/Next Region as Time Selection
	\item \shift \option: Swap Time Selection to left/right
	\item \alt: Previous/Next Region as Loop
	\item \control \alt:  Swap Loop to left/right
\end{itemize}
\end{itemize}

