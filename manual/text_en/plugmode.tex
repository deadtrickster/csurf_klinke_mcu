\subsection{FX-Mode}\label{plugmode} 

The FX-Mode allows you to edit FX parameters. To structure the hugh
amount of parameters that a FX can have, the parameters are divided
into "Banks" and "Pages". Each Page can control up to 16 parameters (8
with the \faders and 8 with the \vpots), and each Bank contains 8
Pages. A map consists of 8 Banks.

The parameters mapped to \faders are controlled as continuous
parameters (like e.g. thresholds, frequencies, etc.), the parameters
mapped to \vpots are handled as discreet parameters with a limited
number of values (e.g.  for On/Off buttons or the selection of a
Waveform). The \flip button bypasses the FX, the master fader controls
the Dry/Wet level.


For many FXs, maps are installed in the MCU subfolder of the \reaper
plugin folder (see~\ref{installation}). In the following description,
these are called ``factory maps''. However, you can also create your
own maps for the selected FX using the \mapeditor ( see section
\ref{createmaps} for details). If no map exists for a FX, a mapping is
created automatically. In this case all parameters are processed as
continuous (so that the \vpots are unused for FXs without user created
maps).


\subsubsection{Navigation}

A Bank is selected by pressing the \solo buttons, a Page by pressing
the \mute buttons. The \solo/\mute LED flashes if the corresponding
Bank/Page is selected, it illuminates, if the Bank/ Page contains parameters
that can be controlled, or is off if neither is the case.

An FX can be selected by pressing the \plug button of the VPOT ASSIGN
section a second time. Then the names of up to six FXs are displeyed
and can be selected by pressing the corresponding \vpot. The two other
\vpots allow you to scroll through the FXs. An FX can also be selected
by pressing the \select buttons. They select one of the first 8 FXs,
or, when pressed in combination with the \shift modifier, one of the
FXs 9-16.

\subsubsection{Favorites}\label{fav}

By default, you can only control an FX of the selected track. If you
select another track, the FX that was selected in this track will get
the focus again. By pressing the \gv button, this connection between
Tracks and controlled FX can be broken, the controlled FX will stay
fixed untill a different FX is selected manually.

In this \gv state it is possible to store and recall an FX as a
favorite. To store the FX press \control \select, to recall it 
\select only (so an FX that is not a favorite can only selected via the
\vpots, see section above). With the \shift button you get eight
additional slots, so you can store up to 16 favorites.

\subsubsection{FX Presets}\label{fxpresets}

The \rec buttons can be used to store FX presets. An FX preset is a
collection of all parameter-values of parameters that are mapped to a
\fader or \vpot plus the dry/wet and bypass setting. The \rec button
define the preset number (e.g. the \rec button of \cs 3 is used to
store or recall preset 3). Presets are bound to an FX instance and can
not copied to a different FX instance.

To store a preset, press \control \rec. As for the
\hyperref[fav]{Favorites} you can use the \shift modifier to store 8
additional presets. To recall a preset press  the \rec button only.
It is possible to store and recall presets for all FXs of a track at
once, to do this  double-click the \rec button. You can use
this to compare settings that involve more than one FX, or to compare
two different FXs by bypassing the first when you store the preset 1, and
bypassing the second FX, when you store preset 2.

The LED of a \rec button is illuminated if a preset is stored in this
position, or it flashes when a preset is stored or recalled without
changing a parameter afterwards.

The \alt modifier allows you to perform blind-tests. As long as you
press the \alt modifier the control surface will not update it's
\faders and LEDs to the selected preset. When you press a \rec button
while holding \alt, a random preset for the FX will be selected (so it
does not matter which \rec button is pressed). Double-clicking a \rec
button also allows you to select a random preset for all FXs of a
track, whereby the preset number will be selected from the existing
presets of the selected FX (if another FX of the track does not have a
preset in this position, it will not change its state).

Presets can be deleted. When you have a lot of presets with many
parameters, this can slow down actions like ``add track''. To delete a
preset, press the \option modifier while pressing the corresponding
\rec button. If you double-click the \rec button, the corresponding
presets of all FXs will be deleted. To delete all presets of the FX
also press the \alt button. Double-clicking the \rec button will
extend this to all tracks, so that all FX presets in the track will be
deleted.

\subsubsection{Map Creation}\label{createmaps}

You can create your own maps can with the
\hyperref[F:Screenshot_Plug_Mode]{Mapping editor} which can be opened
via \alt \plug. The editor has a lot of tooltips. However, an
additional help might be the screencast I made as an example of how to
create a map: \url{http://tinyurl.com/33btx7a}.

\pikWrapFigure[0.54][9.3cm][-2.3mm][0.2mm]{Screenshot_Plug_Mode}{Mapping
  editor (\alt \plug)}{}{}
When an FX is selected via the MCU or the name of the selected FX is
changed in the Track FX dialog, a map is loaded if it's name is part
of the FX name. E.g.  if the FX name is "VST: ReaEQ (Cockos)" and a
map "ReaEQ" exists, this map will be loaded. First the user-created
maps \footnote{Which are stored in the {\tt My
    Documents/MCU/PlugMaps/} folder.} are matched, if no map is found
there, the installed factory maps are searched through. So you can
modify a factory map and store it with the same name, afterwards the
modified version will always be loaded.

A mapping can be stored using the ``save'' button at the top of the editor. To
avoid ambiguity you can not  save a map with a name that contains the
name of an  existing user map or that is part of an existing user map.
The ``save'' dialog will only allow you to save maps with
unambiguous names, but it is best not to rename files in {\tt My
Documents/MCU/PlugMaps/}.

When an FX is inserted or selected, the corresponding map will be
loaded automatically. The FX is matched to the maps using the FX name
in the FX: Track window of \reaper. It is important to keep in mind
that FX maps are not cached in the memory (except
\hyperref[localmaps]{Local maps} which are explained in the next
section). So if you edit a map and then select a different FX, these
changes are lost. When you start to work on a map, it is a good idea
to save an empty map and activate the Autosave option in the Mapping
editor (you must save the map first, so that the Autosave function
knows a filename). When Autosave is enabled, the map will be saved
automatically when you select a different FX.

If a map is named as the FX, no further steps are needed after
inserting the FX, but if you load a plug-in like Reaktor for which you
need different maps depending on the instrument/effect you are loading
in Reaktor, you must add the name of the map to the FX name. In that
case you can save the plug-in/content setup as an FX chain to avoid
this step. An example is given in this screencast:
\url{http://tinyurl.com/35kuglx}

\subsubsection{Local Maps}\label{localmaps}

Normally a saved change in a map will automatically be used for all FX instances
with matching names. This can be a disadvantage if you want different maps for
the same FX, e.g. if you use the IX/Mixer FX and want to rename the
parameters so that they describe the signal, instead of using just
the default ``Level x + y (dB)'' names. The solution is to create a Local map.

You can do this via the \mapeditor. In the mapping area at the top is
a checkbox called ``Local''. If this checkbox is selected, the map of
this FX instance is independent of the map of other instances and
stored in the \reaper project file and not a separate file in the
\verb|PlugMaps| folder.

\subsubsection{Sync with \reaper GUI}\label{plugsync}
The default behaviour for syncing the MCU with the \reaper GUI is as follows: 
\begin{itemize}
\item When you select a different FX in the FX chain of the track that is
selected with the MCU or open a new floating window of a FX from the 
selected track, the MCU also jumps to this FX.\footnote{With the exception 
of when the favorite mode is active (= \gv LED is lit).}
\item When you select a different FX with the MCU and the \reaper FX chain
window of the corresponding track is visible, the FX will also be selected in
the FX chain window. This can be avoided by pressing \option while selecting the FX.
\end{itemize}

\noindent
This behaviour can be changed, see Table~\ref{T:plugmode_options1} and
Table~\ref{T:plugmode_options2} for details.

\vspace{-1mm}
\pikTable[14.505cm]{plugmode_options1}{Main Options in Plug-Mode
  (\option)}{}{}
%\vspace{1mm}
\pikTable[14.485cm]{plugmode_options2}{2nd Options in Plug-Mode (\shift \option)}{}{}


\subsubsection{Summary}

\begin{itemize}
\item \vpot: Control discreet parameters. You can select the next
  value by pressing the \vpot. In combination with \shift you select
  the previous. If it was the highest value, you will jump to the
  lowest (e.g. if only two values exist e.g. Off and On, you will
  toggle between off and on by pressing \vpot). Rotating the \vpot
  works nearly in the same way, but does not jump from the highest to
  the lowest value (so in the example given above you could turn the
  \vpot to the right as long as you wanted, and would get only the On
  state).
\item \rec: Recall an \hyperref[fxpresets]{FX preset}
\bemod
	\begin{compactitem}
	  \item \shift\footnote{All of these \rec modifiers can be combined.}: Access FX
	  preset 9-16
	  \item \control: Store an FX preset
	  \item \option: Delete an FX preset
	  \item \alt: Recall a random FX preset, or in combination with \option: Delete
	  all presets of the FX
	  \item Double-click: Extend action to all FXs of the track
	\end{compactitem}
\item \solo: Select a Bank
\item \mute: Select a Page. 
\bemod
	\begin{compactitem}
	\item \shift In all Banks the Pages at this position are selected 
	\end{compactitem} 
\item \select: Select an FX of the actual track (if the \gv is off) or a
Favorite (else). 
\bemod
	\begin{compactitem}
        \item \control: If Favorites mode is active: store a Favorite.
	 \item \shift: Access FX or Favorite 9-16. 
	\end{compactitem} 
\item \bankdu Select  previous/next Bank
\item \channeldu Select  previous/next Page
\item \gv: see \select
\item \flip: Bypass
\item \mfader: Control Dry/Wet level.
\item Double-click (and hold) \nv: Show all parameter values in the display.
\end{itemize}


%%% Local Variables: 
%%% mode: latex
%%% TeX-master: "../mcu_klinke_manual"
%%% End: 
