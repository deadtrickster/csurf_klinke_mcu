\begin{footnotesize}
\begin{tabular}{@{\extracolsep{\tableColSep}\hspace{\tableColSepHSpace}}ccl}
\optionHeader
\rowcolor{TC1}
MCU follow&
off&
the MCU controlled FX can only be changed via the MCU itself\\
\rowcolor{TC2}
&
same track&
the default behavior as described above\\
\rowcolor{TC1}
&
always&
the MCU always keeps in sync with the last selected FX in the
GUI,\\
\rowcolor{TC1}
&
&
even if the FX is on a different track or the favorite mode is
active\\
\ts
\rowcolor{TC2}
GUI follow&
off&
\mcu never modifies the \reaper GUI\\
\rowcolor{TC1}
&
if chain open&
the default behavior as described above\\
\rowcolor{TC2}
&
open chain&
same as "if chain open", but also opens an FX chain window, if none  exists\footnote{depending on the \reaper properties, this could also close the
"old" FX chain window}\\
\rowcolor{TC1}
&
open floating&
opens a floating window of the selected FX \\
\ts
\rowcolor{TC1}
\alt \select&
open chain&
when you press a \select button while holding \alt, the FX
chain\\
\rowcolor{TC1}
&
&
window will be opened\\
\rowcolor{TC2}
&
-float&
as above, but also all floating FX windows will be closed\\
\rowcolor{TC1}
&
open float&
a floating window of the selected FX will be opened\\
\ts
\rowcolor{TC2}
Limit float&
off&
don't limit the number of floating FX windows\\
\rowcolor{TC1}
&
only 1 MCU&
when you open a second floating window using the MCU, the first
will\\
\rowcolor{TC1}
&
&
be closed, but you can still open additional floating windows via \reaper\\
\rowcolor{TC2}
&
only selected&
only the plug-in that is controlled by the MCU can have a
floating\\ 
\rowcolor{TC2}
&
&
window, all others will be closed.\footnote{this will automatically set the
"MCU follow" option to "always"}\\
\rowcolor{TC1}
&
only chain&
no floating windows can be opened, instead you will always get
a FX\\ 
\rowcolor{TC1}
&
&
chain window\\ 
\tf
\end{tabular}
\end{footnotesize}
